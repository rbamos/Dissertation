\section{Conclusion}
\label{sec:ppot:conclusion}
We developed a system for the longitudinal, large-scale, automated collection and curation of privacy policies. Using the system we built a dataset of over 1M privacy policies that span more than two decades. We found that privacy policies are becoming longer and harder to read.

We developed an automated trend surfacing tool and investigated some of the surfaced trends. Specifically, we investigated trends around third parties, tracking technologies, and self-regulatory bodies. Our results suggest that privacy policies show a concerning lack of transparency: the usage of third parties and tracking technologies is severely underreported.

Our results add to the growing body of evidence suggesting the inadequacy of the ``notice and choice'' model for privacy policy regulation and demonstrate the monumental impact of the GDPR.


\textbf{Release.} Our dataset and source code is available for public use.\footnote{ \url{https://privacypolicies.cs.princeton.edu/}} The access requests we have received have revealed a diversity of use cases for our data, including the study of responses to regulation, automated enforcement, health information technology, and ethics.

In addition to making data and code available, inspired by the TOSBack project~\cite{TOSBack}, we built a chronologically accurate GitHub repository of the policy documents in our dataset.
We used \emph{GitPython}~\cite{GitPython} to commit privacy policy texts and HTML sources (in a separate branch) using the time-of-archive timestamps and detailed privacy policy metadata. 
GitHub’s easy to use web interface enables users without technical skills to perform full text search, compare different versions of the same policy, or use the ``Blame’’ feature to trace modifications.
Finally, we built a separate web interface
to make it easier to search, filter, and explore the privacy policies in the GitHub repository.


\textbf{Limitations.}
Our data collection and analyses were limited to privacy policies; we did not analyze how archived websites track users or share personal data with third parties. 

For internationalized websites, the location of the Wayback Machine crawlers could determine the language of the archived site and its privacy policy. 
Although ``the vast majority of captures are initiated from the [United States],'' there is no way to determine the location, country, or IP address of the Wayback Machine crawlers that captured a particular snapshot~\cite{crawl-location-tweet-by-graham}.


For our readability discussion, we noted that readability metrics have limitations; they do not consider organization, formatting, or length of the document, nor the semantic difficulty of the document. 

In our discussion of how terms trend over time, we used (typically handcrafted) regular expressions to search for referenced terms. Despite our best efforts and extensive manual validation, these regular expressions may capture terms that we did not intend to capture. They additionally may capture terms we intended to capture, but do not fit the targeted concept, and they may miss terms that fit the targeted concept. 


\textbf{Future work.}
Based on our results we suggest several directions for future work. The reporting gap for third parties and tracking technologies could be investigated with a combination of web measurements and privacy policy analysis. Another direction is to further investigate ambiguity in privacy policies.

More sophisticated natural language processing techniques applied to longitudinal privacy policy data could improve our understanding of how online services adapt to new privacy regulations, the prevalence of specific business practices, and whether online services are complying with privacy laws. 
We anticipate that our dataset will prove useful for research in all of these directions.
