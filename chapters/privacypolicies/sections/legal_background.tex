{\section{Legal background}
\label{sec:ppot:background}

In this section, we review the evolving legal context for privacy policies in the United States and the European Union, since we aim to examine how privacy policies respond to legislative and regulatory developments. 

Since the early days of the commercial internet, privacy policies have had a hotly contested role in protecting user information. In the United States, voluntary disclosures about data practices by companies in their privacy policies are the foundation of the federal ``notice and choice'' approach to consumer privacy. The European Union has taken a different direction by specifying what should be included in privacy disclosures as part of its more comprehensive approach to data protection regulation. 

\textbf{Privacy policies in the United States: notice and choice.} In the mid-1990s, U.S. policymakers faced a fundamental question about how to regulate privacy online. They could directly regulate data practices, or they could leave privacy protections to the nascent online business ecosystem. Given evidence that market forces were failing to protect privacy and that this might be affecting economic growth, policymakers landed on a light-touch hybrid model. Online services would disclose their privacy practices to consumers on a mostly voluntary basis, then the Federal Trade Commission (FTC) would police those disclosures for accuracy~\cite{swire1997markets}. The FTC’s enforcement actions, which usually result in settlements with consent decrees, would provide guidance to the private sector about best practices and develop a common law of online privacy~\cite{solove2014ftc}.

The underlying theory behind the policy rested on ``the fundamental precepts of awareness and choice''~\cite{united1997framework}. Specifically, ``Data-gatherers should inform consumers what information they are collecting, and how they intend to use such data; and Data-gatherers should provide consumers with a meaningful way to limit use and re-use of personal information''~\cite{united1997framework}. Lurking in the background was the threat that policymakers would step in with regulation if the light-touch model failed. Additionally, Congress enacted legislation to require disclosures about data practices (and sometimes provide individual privacy rights) in specific sectors, including healthcare (the Health Insurance Portability and Accountability Act) and finance (the Gramm-Leach-Bliley Act).

\textbf{Children's Online Privacy Protection Act.} Children's privacy was the first internet-specific area to be directly regulated in the United States. In 1998, the FTC issued a study of children's privacy, which led it to recommend legislation that would place parents in control of their children's information. COPPA was enacted that year and regulates the collection and use of information of websites directed at children under 13 years of age. COPPA’s passage and the threat of further regulation led to the adoption of privacy policies on many popular websites.

\textbf{California Online Privacy Protection Act.} California introduced legislation that requires commercial websites to post privacy policies~\cite{caloppa}. While CalOPPA is a state law, after it came into force in 2004 it quickly became a de facto national privacy policy requirement. The recent California Consumer Privacy Act and California Privacy Rights Act~\cite{ccpaANDcpra} represent a break from the notice and choice model, guaranteeing consumers specific privacy rights.

\textbf{Privacy policies in the European Union: data protection.}
Europe has taken a more direct approach to privacy regulation, specifying the information that businesses must disclose about their privacy practices and guaranteeing individual privacy rights.

\textbf{Data Protection Directive.} The DPD, which went into effect in 1998, required EU member states to establish comprehensive privacy regimes that included disclosures, opt-out rights, and specific protections for sensitive data, including information about religious beliefs, sexual orientation, medical history, and financial circumstances. These rights were enforced by specialized regulatory agencies, data protection authorities, in the member states.

\textbf{General Data Protection Regulation.} In 2016, the EU overhauled its approach to protecting privacy by enacting the GDPR. The GDPR includes a comprehensive slate of privacy disclosure and control requirements, and it opens the door to serious penalties for violations. Most relevant to our work, the GDPR mandates a number of specific disclosures about how firms process individual data and how individuals can exercise their privacy rights. Online services typically provide these disclosures in their privacy policies.

}