\chapter{Introduction} \label{ch:introduction}

The rise of the internet has introduced new challenges in consumer protection. Cheap and efficient data collection has introduced novel privacy risks. Low service costs have lead to novel profit collection methods, including an amazing variety of free services. Global marketplaces have lead to large drops in prices as online platforms reduce market friction and increase competition. The internet has given an opportunity to amplify the voices of many. The intense competition and the power of speech on the internet has created a strong incentive to engage in dishonest, self-serving speech. The very anonymity that protects those voices also protects the voices of those seeking to be dishonest.

Consumer protection is an important issue. Many jurisdictions ranging from local to national have passed consumer protection laws. In the United States, the primary national body governing consumer protect is the Federal Trade Commission (FTC), established by the Federal Trade Commission Act of 1914 \raNote{citeme}, and many states have their own entities focused on consumer protection. The FTC has highlighted a number of issues at the intersection of the internet and consumer protection including fair reviews and adequate disclosure of privacy practices.

Since 1914, advancements in communications technology have introduced novel risks to consumers. 
- Person-to-person, the Archimedes principle ~\cite{thompson2008archimedes}
- Postal mail scams ~\cite{uspismailfraud}
- Phone scamming? ~\cite{ftcphonescams}
- Television's influence on children's diets ~\cite{morton1985television}
- The introduction of the internet has lead to a host of new consumer protection issues, ranging from poor privacy protections, new types of fraud, sponsorship disclosures, and 

There is a rich literature of consumer protection work on the internet, which we explore in greater depth in Chapter \ref{ch:background}.

Consumer protection work
- Consumer protection issues on the web
- Why does consumer protection matter?
- Quick outline of prior work
- Common theme: single point in time
- Our contribution: longitudinal analysis
- Specific breakdown of the contributions

Prior work analyzing changes in corpuses? Prior work analyzing changes in recommender systems?



\section{Structure}

In this work, our contributions are two large-scale, longitudinal web crawls and corresponding analyses studying consumer protection issues. 

We will explore background material on consumer protection in Chapter \ref{ch:background}. In Chapter \ref{ch:ppot} we present our work on longitudinal collection and analysis of privacy policies. \raNote{Something about 1M policies, some key findings, some takeaways} We will switch gears to our work on longitudinal collection and analysis of reviews on Yelp in Chapter \ref{ch:rim}. \raNote{Something about 12M reviews, some key findings, some takeaways} Finally, we'll discuss takeaways and future work in Chapter \ref{ch:conclusion}. \raNote{flesh this out a little more. Give some ``sneak peeks''}