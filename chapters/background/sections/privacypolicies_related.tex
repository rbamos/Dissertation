\section{Privacy policies}
\label{sec:ppot:related}

%\raNote{Mesh together}

As the study of privacy policies is the focus of Chapter \ref{ch:ppot}, we go deeper in depth on the literature surrounding privacy policies. Increasing adoption of privacy policies and changes in the regulatory environment, such as the changes outline in Section \ref{sec:ppot:background}, have led to a rich research literature. We highlight marketplace studies and other longitudinal studies, other research datasets, compliance checking approaches, studies of consumer comprehension, and applications of machine learning to privacy policies.

\textbf{Marketplace and longitudinal studies.}
The earliest work on privacy policies consisted of marketplace surveys. A sequence of studies by the FTC~\cite{ftc-privacy-survey1998, ftc-privacy-survey2000} and Culnan~\cite{culnan2000protecting}, from 1998 to 2000, found that U.S. websites were rapidly adopting privacy policies but that the content of policies was often spotty. In the U.K., a 2002 Information Commissioner's Office (ICO) study reported similar results~\cite{ico-survey2002}. More recent surveys have called attention to privacy policy shortcomings in specific sectors, such as healthcare~\cite{sunyaev15} and finance~\cite{bowers2017}. In a market survey concurrent to this project, Srinath et al. report readability scores, topic models, key phrases, and textual similarity for a corpus of just over a million privacy policies~\cite{srinath2020}.

Several projects have expressly incorporated a longitudinal dimension. Milne and Culnan compared data from a set of marketplace surveys between 1998 and 2001, concluding that privacy policy adoption was increasing and that policies were including greater disclosures about collection, sharing, choice, security, and cookies~\cite{milne2002}. Later, Milne et al. compared a sample of privacy policies from 2001 and 2003, finding that readability was decreasing and length was increasing~\cite{milne2006longitudinal}. Antón et al. examined a small number of healthcare privacy policies between 2000 and 2003, observing similarly decreased readability and greater disclosures~\cite{anton2007hipaa}.

Recent work has used longitudinal privacy policy analysis to examine the effects of the GDPR. Degeling et al. used crawl data from 2018 and Wayback Machine data from 2016 and 2017 to examine privacy policies on over 6,000 websites before and after the GDPR took effect; the findings include a slight uptick in privacy policy adoption, increased use of key phrases related to the GDPR, and inconsistent privacy policy update practices~\cite{degeling2018we}. Linden et al. evaluated a privacy policy corpus of similar size, using Wayback Machine snapshots from 2016 and 2019, and found that privacy policies were longer, more likely to include categories of disclosures, and generally included more specific disclosures~\cite{linden2020privacy}.

We contribute to this area of literature with a longitudinal analysis of much larger scope, both in time (spanning over two decades) and in number of websites (over 130,000). We characterize longer-term document-level trends than the prior work, characterize trends by website popularity, and contextualize shifts associated with the GDPR (Section \ref{sec:ppot:doclevstatstime}). We also provide the first longitudinal analysis of specific trends (Section \ref{sec:ppot:analysis}).

\textbf{Research datasets.}
Another thread in the privacy policy literature is developing privacy policy research datasets to enable future study. Ramanath et al. contributed the earliest dataset in 2014, a collection of over 1,000 manually segmented privacy policies~\cite{ramanath2014unsupervised}. In 2016, Wilson et al. released OPP-115, a set of 115 manually annotated website privacy policies~\cite{wilson2016creation}. Companion work by Wilson et al. demonstrated the feasibility of crowdsourcing to annotate a privacy policy dataset~\cite{wilson2016crowdsourcing}. In 2019, Zimmeck et al. contributed APP-350, a dataset of 350 annotated mobile app privacy policies, and MAPS, a dataset of nearly 450,000 app privacy policy URLs~\cite{zimmeck2019}. Concurrent to this work, Srinath et al. released PrivaSeer, a dataset of over 1 million English privacy policies extracted from May 2019 Common Crawl data~\cite{srinath2020}. Also concurrent to this work, Zaeem et al. contributed a dataset of hundreds of thousands of privacy policies with associated website categories~\cite{zaeemlarge}.

Our work advances this area of the privacy policy literature by providing what is, to our knowledge, the first large-scale longitudinal dataset. We facilitate future work on changes over time to privacy policies by applying new techniques not only to current privacy policies but also to our curated dataset of historical policies.

\textbf{Compliance checking.}
A separate strand of research has examined whether privacy policies are in compliance with legal requirements and whether policies fully disclose data practices (regardless of legal requirements). Several projects have compared privacy policies to data flows and app permissions, noting pervasive gaps in disclosures~\cite{slavin2016, zimmeck2019}. Linden et al. compared pre- and post-GDPR privacy policies to ICO guidance and found improving though inconsistent compliance~\cite{linden2020privacy}. In the compliance work most similar to our own, Marotta-Wurgler examined nearly 250 privacy policies in 2016 for claims of compliance with regulatory and self-regulatory programs~\cite{marotta-wurgler2016}.


We contribute to the privacy policy compliance literature by evaluating representations about regulatory and self-regulatory programs over time. We identify the long-term rise of specific self-regulatory programs,
 and we identify shifting representations about regulatory programs in response to legal developments
(Section \ref{subsec:ppot:trends}). We also compare aggregate privacy policy disclosures about advanced tracking technologies to the known level of adoption of those technologies (Section \ref{subsec:ppot:trends}).

\textbf{Consumer comprehension.}
Another line of privacy policy research examines consumer comprehension. Surveys and intervention studies have demonstrated that consumers inconsistently read privacy policies and primarily read privacy policies to exercise control~\cite{milne2004}; consumer trust in websites has little connection to their privacy policies~\cite{pan2006}; consumers have limited comprehension of privacy policy content~\cite{vail2008, mcdonald2009comparative}; consumers interpret privacy policies differently from legal experts~\cite{reidenberg2016disagreeable, strahilevitz2016}; and consumers have difficulty exercising privacy choices in privacy policies~\cite{habib2020}. Studies have repeatedly shown that privacy policies are lengthy and difficult to read~\cite{jensen2004, mcdonald2008cost, li2012online, fabian2017large}. While the precise readability metrics vary by project, recent work has demonstrated that these metrics are strongly correlated for privacy policies~\cite{fabian2017large}. Textual analysis has also highlighted the prevalence of ambiguous claims in privacy policies~\cite{reidenberg2016ambiguity}.

We advance the literature on consumer comprehension of privacy policies by tracking how length and readability have evolved over two decades and across website ranking tiers.

\textbf{Applied machine learning.}
A final related area of the privacy policy literature applies machine learning techniques to 
privacy policy texts. Recent work has identified specific claims in policies and assigned overall grades~\cite{zimmeck2014}, identified choices provided in privacy policies~\cite{sathyendra2017, kumar2020}, and enabled structured and free-form queries about privacy policy attributes~\cite{harkous2018polisis}. While our own privacy policy analysis relies on heuristics rather than  machine learning, the dataset we contribute is intended to enable future machine learning projects to easily present longitudinal results.