

With the development of any new communications technology, new threats to consumers have emerged. As the internet has matured, it too has become host to numerous hazards for consumers. In this work, we study the impact of internet systems on two consumer protection issues: consumers' right to privacy and consumers' right to be informed. We present two studies: one of website privacy policies and one of reviews on yelp. By taking a longitudinal approach, we demonstrate that consumer protection issues are moving targets -- these issues exist in an evolving landscape. 

In our work on privacy policies, we collected and curated a dataset of over 1M privacy policies. Our analyses of the data paint a troubling picture of the transparency and accessibility of privacy policies. By comparing the occurrence of tracking-related terminology in our dataset to prior web privacy measurements, we find that privacy policies have consistently failed to disclose the presence of common tracking technologies and third parties. We also find that over the last twenty years privacy policies have become even more difficult to read, doubling in length and increasing a full grade in the median reading level. Our data indicate that self-regulation for first-party websites has stagnated, while self-regulation for third parties has increased but is dominated by online advertising trade associations. Finally, we contribute to the literature on privacy regulation by demonstrating the historic impact of the GDPR on privacy policies.

In our work on reviews, we collected three datasets, each representing a unique cross-section of the review landscape, totalling over 12.5M reviews. Our first dataset provides an eight year comparison, our second provides finer grained comparison over one year, concentrated in a single region, and our third provides a four month perspective of a random sampling of zipcodes, stratified by both population density and income. We focus our analysis primiarly on ``reclassification,'' wherein a platform changes its filtering decision for a review. We find that reviews routinely move between Yelp's two main classifier classes (``Recommended'' and ``Not Recommended''), and authors' reviews tend to move as a group between classes. Some reviews move multiple times: we observed up to five reclassifications in eleven months. Our data suggests demographic disparities in reclassifications, with more changes in lower density and low-middle income areas. Because our web crawls coincided with the COVID-19 pandemic, our data also allowed limited exploration of the impact of mask policies and discussions on reviews.
