% Word limit: 350 words https://library.princeton.edu/special-collections/sites/default/files/Dissertation%20Formatting%20Requirements.pdf

Consumer protection issues date back to antiquity, with new technologies like the world wide web have introduced new hazards for consumers. While many academics have studied consumer protection on the web at various points in time, as a dynamic environment, the study of the web demands a longitudinal perspective. In this work, we examine how consumers' right to privacy and consumers' right to be informed have been impacted by the web. Our work highlights the key role in study of consumer protection issues played by longitudinal analyses and longitudinal data collection -- data collected over repeated, time-spaced passes.

We explore consumer protection issues on the web through two lenses: longitudinal study of website privacy policies and longitudinal study of reviews on Yelp. We approach both problems first by collecting data with automated, repeated visits to the websites of interest to collect large scale datasets. In our study of privacy policies, we aggregate Internet Archive's crawls to perform longitudinal collection, whereas in our online reviews study, we crawl the data ourselves. We collected over 1 million privacy policies over 22 years and over 12M reviews over 11 months. 

In our study of privacy policies, we examine how privacy policies have evolved. We find gaps in disclosure of privacy-related practices. We show declining readability over the long term, doubling in length and becoming more complex. Our work helps illustrate some of the historic impacts of the GDPR. In our study of online reviews, we present the first study of ``reclassification,'' wherein a platform changes its filtering decision for a review. We find that reviews routinely move between Yelp's two main classifier classes (``Recommended'' and ``Not Recommended''), up to five reclassifications on a single review. We identify demographic disparities in review prevalence and filtering decisions.

Our work emphasizes the importance of longitudinal study for consumer protection issues online. We show phenomena that cannot be studied without longitudinal data collection and analysis. We help lay the groundwork for future work on these issues through our software and data releases, easing the pathway for future researchers looking to study such issues.


% In our work on privacy policies, we collected and curated a dataset of over 1M privacy policies. Our analyses of the data paint a troubling picture of the transparency and accessibility of privacy policies. By comparing the occurrence of tracking-related terminology in our dataset to prior web privacy measurements, we find that privacy policies have consistently failed to disclose the presence of common tracking technologies and third parties. We also find that over the last twenty years privacy policies have become even more difficult to read, doubling in length and increasing a full grade in the median reading level. Our data indicate that self-regulation for first-party websites has stagnated, while self-regulation for third parties has increased but is dominated by online advertising trade associations. Finally, we contribute to the literature on privacy regulation by demonstrating the historic impact of the GDPR on privacy policies.

% In our work on reviews, we collected three datasets, each representing a unique cross-section of the review landscape, totalling over 12.5M reviews. Our first dataset provides an eight year comparison, our second provides finer grained comparison over one year, concentrated in a single region, and our third provides a four month perspective of a random sampling of zipcodes, stratified by both population density and income. We focus our analysis primarily on ``reclassification,'' wherein a platform changes its filtering decision for a review. We find that reviews routinely move between Yelp's two main classifier classes (``Recommended'' and ``Not Recommended''), and authors' reviews tend to move as a group between classes. Some reviews move multiple times: we observed up to five reclassifications in eleven months. Our data suggests demographic disparities in reclassifications, with more changes in lower density and low-middle income areas. Because our web crawls coincided with the COVID-19 pandemic, our data also allowed limited exploration of the impact of mask policies and discussions on reviews.

% \raNote{
% Make sure the structure is good -- try to make it punchy
% First two sentences are vague, try to make it punchy
% To big of a jump from 2 to 3
% Mimic structure of the introduction -- first why longitudinal approach is important before saying we take a longitudinal approach

% e.g. like the 1.1.1 sentence "some consumer protection must be studied..."

% Para 3 needs to focus more on the bigger picture (e.g. the new ideas we've had since)

% Ending paragraph on impact (e.g. data release, but other things). Try to end on big picture

% No major change -- just tweak
% }